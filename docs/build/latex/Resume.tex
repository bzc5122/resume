% Generated by Sphinx.
\def\sphinxdocclass{report}
\documentclass[letterpaper,10pt,english]{sphinxmanual}
\usepackage[utf8]{inputenc}
\DeclareUnicodeCharacter{00A0}{\nobreakspace}
\usepackage[T1]{fontenc}
\usepackage{babel}
\usepackage{times}
\usepackage[Bjarne]{fncychap}
\usepackage{longtable}
\usepackage{sphinx}
\usepackage{multirow}


\title{Resume Documentation}
\date{April 16, 2014}
\release{0.0.1}
\author{David Pierson Bradway}
\newcommand{\sphinxlogo}{}
\renewcommand{\releasename}{Release}
\makeindex

\makeatletter
\def\PYG@reset{\let\PYG@it=\relax \let\PYG@bf=\relax%
    \let\PYG@ul=\relax \let\PYG@tc=\relax%
    \let\PYG@bc=\relax \let\PYG@ff=\relax}
\def\PYG@tok#1{\csname PYG@tok@#1\endcsname}
\def\PYG@toks#1+{\ifx\relax#1\empty\else%
    \PYG@tok{#1}\expandafter\PYG@toks\fi}
\def\PYG@do#1{\PYG@bc{\PYG@tc{\PYG@ul{%
    \PYG@it{\PYG@bf{\PYG@ff{#1}}}}}}}
\def\PYG#1#2{\PYG@reset\PYG@toks#1+\relax+\PYG@do{#2}}

\expandafter\def\csname PYG@tok@gd\endcsname{\def\PYG@tc##1{\textcolor[rgb]{0.63,0.00,0.00}{##1}}}
\expandafter\def\csname PYG@tok@gu\endcsname{\let\PYG@bf=\textbf\def\PYG@tc##1{\textcolor[rgb]{0.50,0.00,0.50}{##1}}}
\expandafter\def\csname PYG@tok@gt\endcsname{\def\PYG@tc##1{\textcolor[rgb]{0.00,0.27,0.87}{##1}}}
\expandafter\def\csname PYG@tok@gs\endcsname{\let\PYG@bf=\textbf}
\expandafter\def\csname PYG@tok@gr\endcsname{\def\PYG@tc##1{\textcolor[rgb]{1.00,0.00,0.00}{##1}}}
\expandafter\def\csname PYG@tok@cm\endcsname{\let\PYG@it=\textit\def\PYG@tc##1{\textcolor[rgb]{0.25,0.50,0.56}{##1}}}
\expandafter\def\csname PYG@tok@vg\endcsname{\def\PYG@tc##1{\textcolor[rgb]{0.73,0.38,0.84}{##1}}}
\expandafter\def\csname PYG@tok@m\endcsname{\def\PYG@tc##1{\textcolor[rgb]{0.13,0.50,0.31}{##1}}}
\expandafter\def\csname PYG@tok@mh\endcsname{\def\PYG@tc##1{\textcolor[rgb]{0.13,0.50,0.31}{##1}}}
\expandafter\def\csname PYG@tok@cs\endcsname{\def\PYG@tc##1{\textcolor[rgb]{0.25,0.50,0.56}{##1}}\def\PYG@bc##1{\setlength{\fboxsep}{0pt}\colorbox[rgb]{1.00,0.94,0.94}{\strut ##1}}}
\expandafter\def\csname PYG@tok@ge\endcsname{\let\PYG@it=\textit}
\expandafter\def\csname PYG@tok@vc\endcsname{\def\PYG@tc##1{\textcolor[rgb]{0.73,0.38,0.84}{##1}}}
\expandafter\def\csname PYG@tok@il\endcsname{\def\PYG@tc##1{\textcolor[rgb]{0.13,0.50,0.31}{##1}}}
\expandafter\def\csname PYG@tok@go\endcsname{\def\PYG@tc##1{\textcolor[rgb]{0.20,0.20,0.20}{##1}}}
\expandafter\def\csname PYG@tok@cp\endcsname{\def\PYG@tc##1{\textcolor[rgb]{0.00,0.44,0.13}{##1}}}
\expandafter\def\csname PYG@tok@gi\endcsname{\def\PYG@tc##1{\textcolor[rgb]{0.00,0.63,0.00}{##1}}}
\expandafter\def\csname PYG@tok@gh\endcsname{\let\PYG@bf=\textbf\def\PYG@tc##1{\textcolor[rgb]{0.00,0.00,0.50}{##1}}}
\expandafter\def\csname PYG@tok@ni\endcsname{\let\PYG@bf=\textbf\def\PYG@tc##1{\textcolor[rgb]{0.84,0.33,0.22}{##1}}}
\expandafter\def\csname PYG@tok@nl\endcsname{\let\PYG@bf=\textbf\def\PYG@tc##1{\textcolor[rgb]{0.00,0.13,0.44}{##1}}}
\expandafter\def\csname PYG@tok@nn\endcsname{\let\PYG@bf=\textbf\def\PYG@tc##1{\textcolor[rgb]{0.05,0.52,0.71}{##1}}}
\expandafter\def\csname PYG@tok@no\endcsname{\def\PYG@tc##1{\textcolor[rgb]{0.38,0.68,0.84}{##1}}}
\expandafter\def\csname PYG@tok@na\endcsname{\def\PYG@tc##1{\textcolor[rgb]{0.25,0.44,0.63}{##1}}}
\expandafter\def\csname PYG@tok@nb\endcsname{\def\PYG@tc##1{\textcolor[rgb]{0.00,0.44,0.13}{##1}}}
\expandafter\def\csname PYG@tok@nc\endcsname{\let\PYG@bf=\textbf\def\PYG@tc##1{\textcolor[rgb]{0.05,0.52,0.71}{##1}}}
\expandafter\def\csname PYG@tok@nd\endcsname{\let\PYG@bf=\textbf\def\PYG@tc##1{\textcolor[rgb]{0.33,0.33,0.33}{##1}}}
\expandafter\def\csname PYG@tok@ne\endcsname{\def\PYG@tc##1{\textcolor[rgb]{0.00,0.44,0.13}{##1}}}
\expandafter\def\csname PYG@tok@nf\endcsname{\def\PYG@tc##1{\textcolor[rgb]{0.02,0.16,0.49}{##1}}}
\expandafter\def\csname PYG@tok@si\endcsname{\let\PYG@it=\textit\def\PYG@tc##1{\textcolor[rgb]{0.44,0.63,0.82}{##1}}}
\expandafter\def\csname PYG@tok@s2\endcsname{\def\PYG@tc##1{\textcolor[rgb]{0.25,0.44,0.63}{##1}}}
\expandafter\def\csname PYG@tok@vi\endcsname{\def\PYG@tc##1{\textcolor[rgb]{0.73,0.38,0.84}{##1}}}
\expandafter\def\csname PYG@tok@nt\endcsname{\let\PYG@bf=\textbf\def\PYG@tc##1{\textcolor[rgb]{0.02,0.16,0.45}{##1}}}
\expandafter\def\csname PYG@tok@nv\endcsname{\def\PYG@tc##1{\textcolor[rgb]{0.73,0.38,0.84}{##1}}}
\expandafter\def\csname PYG@tok@s1\endcsname{\def\PYG@tc##1{\textcolor[rgb]{0.25,0.44,0.63}{##1}}}
\expandafter\def\csname PYG@tok@gp\endcsname{\let\PYG@bf=\textbf\def\PYG@tc##1{\textcolor[rgb]{0.78,0.36,0.04}{##1}}}
\expandafter\def\csname PYG@tok@sh\endcsname{\def\PYG@tc##1{\textcolor[rgb]{0.25,0.44,0.63}{##1}}}
\expandafter\def\csname PYG@tok@ow\endcsname{\let\PYG@bf=\textbf\def\PYG@tc##1{\textcolor[rgb]{0.00,0.44,0.13}{##1}}}
\expandafter\def\csname PYG@tok@sx\endcsname{\def\PYG@tc##1{\textcolor[rgb]{0.78,0.36,0.04}{##1}}}
\expandafter\def\csname PYG@tok@bp\endcsname{\def\PYG@tc##1{\textcolor[rgb]{0.00,0.44,0.13}{##1}}}
\expandafter\def\csname PYG@tok@c1\endcsname{\let\PYG@it=\textit\def\PYG@tc##1{\textcolor[rgb]{0.25,0.50,0.56}{##1}}}
\expandafter\def\csname PYG@tok@kc\endcsname{\let\PYG@bf=\textbf\def\PYG@tc##1{\textcolor[rgb]{0.00,0.44,0.13}{##1}}}
\expandafter\def\csname PYG@tok@c\endcsname{\let\PYG@it=\textit\def\PYG@tc##1{\textcolor[rgb]{0.25,0.50,0.56}{##1}}}
\expandafter\def\csname PYG@tok@mf\endcsname{\def\PYG@tc##1{\textcolor[rgb]{0.13,0.50,0.31}{##1}}}
\expandafter\def\csname PYG@tok@err\endcsname{\def\PYG@bc##1{\setlength{\fboxsep}{0pt}\fcolorbox[rgb]{1.00,0.00,0.00}{1,1,1}{\strut ##1}}}
\expandafter\def\csname PYG@tok@kd\endcsname{\let\PYG@bf=\textbf\def\PYG@tc##1{\textcolor[rgb]{0.00,0.44,0.13}{##1}}}
\expandafter\def\csname PYG@tok@ss\endcsname{\def\PYG@tc##1{\textcolor[rgb]{0.32,0.47,0.09}{##1}}}
\expandafter\def\csname PYG@tok@sr\endcsname{\def\PYG@tc##1{\textcolor[rgb]{0.14,0.33,0.53}{##1}}}
\expandafter\def\csname PYG@tok@mo\endcsname{\def\PYG@tc##1{\textcolor[rgb]{0.13,0.50,0.31}{##1}}}
\expandafter\def\csname PYG@tok@mi\endcsname{\def\PYG@tc##1{\textcolor[rgb]{0.13,0.50,0.31}{##1}}}
\expandafter\def\csname PYG@tok@kn\endcsname{\let\PYG@bf=\textbf\def\PYG@tc##1{\textcolor[rgb]{0.00,0.44,0.13}{##1}}}
\expandafter\def\csname PYG@tok@o\endcsname{\def\PYG@tc##1{\textcolor[rgb]{0.40,0.40,0.40}{##1}}}
\expandafter\def\csname PYG@tok@kr\endcsname{\let\PYG@bf=\textbf\def\PYG@tc##1{\textcolor[rgb]{0.00,0.44,0.13}{##1}}}
\expandafter\def\csname PYG@tok@s\endcsname{\def\PYG@tc##1{\textcolor[rgb]{0.25,0.44,0.63}{##1}}}
\expandafter\def\csname PYG@tok@kp\endcsname{\def\PYG@tc##1{\textcolor[rgb]{0.00,0.44,0.13}{##1}}}
\expandafter\def\csname PYG@tok@w\endcsname{\def\PYG@tc##1{\textcolor[rgb]{0.73,0.73,0.73}{##1}}}
\expandafter\def\csname PYG@tok@kt\endcsname{\def\PYG@tc##1{\textcolor[rgb]{0.56,0.13,0.00}{##1}}}
\expandafter\def\csname PYG@tok@sc\endcsname{\def\PYG@tc##1{\textcolor[rgb]{0.25,0.44,0.63}{##1}}}
\expandafter\def\csname PYG@tok@sb\endcsname{\def\PYG@tc##1{\textcolor[rgb]{0.25,0.44,0.63}{##1}}}
\expandafter\def\csname PYG@tok@k\endcsname{\let\PYG@bf=\textbf\def\PYG@tc##1{\textcolor[rgb]{0.00,0.44,0.13}{##1}}}
\expandafter\def\csname PYG@tok@se\endcsname{\let\PYG@bf=\textbf\def\PYG@tc##1{\textcolor[rgb]{0.25,0.44,0.63}{##1}}}
\expandafter\def\csname PYG@tok@sd\endcsname{\let\PYG@it=\textit\def\PYG@tc##1{\textcolor[rgb]{0.25,0.44,0.63}{##1}}}

\def\PYGZbs{\char`\\}
\def\PYGZus{\char`\_}
\def\PYGZob{\char`\{}
\def\PYGZcb{\char`\}}
\def\PYGZca{\char`\^}
\def\PYGZam{\char`\&}
\def\PYGZlt{\char`\<}
\def\PYGZgt{\char`\>}
\def\PYGZsh{\char`\#}
\def\PYGZpc{\char`\%}
\def\PYGZdl{\char`\$}
\def\PYGZhy{\char`\-}
\def\PYGZsq{\char`\'}
\def\PYGZdq{\char`\"}
\def\PYGZti{\char`\~}
% for compatibility with earlier versions
\def\PYGZat{@}
\def\PYGZlb{[}
\def\PYGZrb{]}
\makeatother

\begin{document}

\maketitle
\tableofcontents
\phantomsection\label{index::doc}



\chapter{David Pierson Bradway}
\label{resume:resume-cv-of-david-pierson-bradway}\label{resume::doc}\label{resume:david-pierson-bradway}
\begin{DUlineblock}{0em}
\item[] \href{mailto:david.bradway@gmail.com}{david.bradway@gmail.com}
\item[] Birthplace: Canton, Ohio
\item[] Birthdate: 1 February 1982
\end{DUlineblock}


\section{Objective}
\label{resume:objective}\begin{itemize}
\item {} 
Career in research, visualization, data acquisition, and signal
processing

\item {} 
Engineering, research and development role in academia or industry,
Autumn 2014

\end{itemize}


\section{Work Experience}
\label{resume:work-experience}\begin{itemize}
\item {} 
\textbf{Technical University of Denmark (DTU)} (Kongens Lyngby, Denmark)

Postdoctoral Researcher, 2013 - present
\begin{itemize}
\item {} 
Developed OpenCL software for processing 3-D Doppler ultrasound
data on the GPU

\item {} 
Conference presentaion, poster, abstracts, and proceedings
accepted

\item {} 
Pursuing pre-clinical feasibility study and peer-reviewed article

\end{itemize}

\item {} 
\textbf{Duke University} (Durham, NC, USA)

Graduate Research and Teaching Assistant, 2005 - 2013
\begin{itemize}
\item {} 
PhD project using ultrasound to noninvasively measure the heart's
mechanical properties

\item {} 
Reviewed scientific literature, formulated and carried out
research plan

\item {} 
Organized and conducted out pre-clinical trials at Duke University
Medical Center

\item {} 
Presented results at conferences, published proceedings and
co-authored articles

\end{itemize}

\item {} 
\textbf{Siemens Healthcare} (Issaquah, WA, USA)

Graduate Student Research Intern, 2008
\begin{itemize}
\item {} 
Worked within a research team in a multinational corporation

\item {} 
Developed feature for research mode of Acuson S2000 ultrasound
scanner

\item {} 
Learned version control and automated build systems

\end{itemize}

\end{itemize}


\section{Education}
\label{resume:education}\begin{itemize}
\item {} 
\textbf{Duke University} (Durham, NC, USA)

Ph.D. in Biomedical Engineering, May 2013.

\item {} 
\textbf{The Ohio State University (OSU)} (Columbus, OH, USA)

B.S. in Electrical and Computer Engineering, June 2005.

\end{itemize}


\section{Relevant Course Work}
\label{resume:relevant-course-work}\begin{itemize}
\item {} 
Digital Signal Processing

\item {} 
Circuits and Instrumentation

\item {} 
Image Processing and Analysis

\item {} 
Systems and Signals

\item {} 
Statistical Signal Processing

\item {} 
C/C++ Programming

\item {} 
Education and communication courses

\end{itemize}


\section{Honors and Activities}
\label{resume:honors-and-activities}\begin{itemize}
\item {} 
Whitaker International Program Scholar (2013)

\item {} 
National Science Foundation Graduate Research Fellow (2005-2008)

\item {} 
Goldwater Research Scholar (2004-2005)

\item {} 
Founded engineering community service group at Ohio State (2003)

\item {} 
Organized engineering design and build trip to Honduran orphanage
(2004)

\end{itemize}


\section{Skills}
\label{resume:skills}\begin{itemize}
\item {} 
Expert in signal and imaging processing programming: Matlab, Python,
LabVIEW

\item {} 
Working knowlegde of other tools and languages: C/C++, OpenCL, R,
Mathematica, MS Office

\item {} 
Picked up for small web projects: PHP, Ruby/Rails, Perl, flavors of
SQL, HTML5, Javascript, Git, and RST

\item {} 
Strong focus on problem solving, signal and image analysis,
scientific computing, and experimental design

\item {} 
Self-motivated execution of a high-level plan with nominal oversight

\item {} 
Strong written and verbal communication, and data visualization
display skills

\item {} 
Successful writer of fellowships, scholarships, and grants

\end{itemize}


\section{Interests}
\label{resume:interests}\begin{itemize}
\item {} 
Tracking Energy efficiency: TED5000 owner, Plotwatt user, Neurio
backer, MS Hohm \& Google PowerMeter ex-user

\item {} 
Creating tools to close feedback loops: measure data, effect change,
and automate it

\item {} 
Personal `hacking' in mobile and embedded systems: Arduino, Raspberry
Pi, Android

\item {} 
Understanding behavior and desicision making: Behavioral Economics,
the Nudge Unit, Dan Ariely's work

\end{itemize}


\section{Publications}
\label{resume:publications}

\subsection{Journal Articles}
\label{resume:journal-articles}\begin{itemize}
\item {} 
BJ Fahey, RC Nelson, DP Bradway, SJ Hsu, DM Dumont, GE Trahey. In
vivo visualization of abdominal malignancies with acoustic radiation
force elastography. Phys Med Biol. 2008 Jan; 53(1):279-93.

\item {} 
BJ Fahey, RC Nelson, SJ Hsu, DP Bradway, DM Dumont, GE Trahey. In
vivo guidance and assessment of liver radio-frequency ablation with
acoustic radiation force elastography. Ultrasound Med Biol. 2008 Oct;
34(10):1590-1603.

\item {} 
KR Nightingale, ML Palmeri, L Zhai, KD Frinkley, M Wang, JJ Dahl, BJ
Fahey, SJ Hsu, DP Bradway, GE Trahey. Impulsive acoustic radiation
force: imaging approaches and clinical applications. The Journal of
the Acoustical Society of America, 2008. vol. 123, issue 5, p. 3792.

\item {} 
KR Nightingale, ML Palmeri, JJ Dahl, DP Bradway, SJ Hsu, RR Bouchard,
SJ Rosenzweig, V Rotemberg, M Wang, L Zhai. Elasticity Imaging with
Acoustic Radiation Force: Methods and Clinical Applications. Japanese
journal of medical ultrasonics. 36. 116, 2009.

\item {} 
PD Wolf, SA Eyerly, DP Bradway, DM Dumont, TD Bahnson, KR
Nightingale, and GE Trahey. Near real time evaluation of cardiac
radiofrequency ablation lesions with intracardiac echocardiography
based acoustic radiation force impulse imaging. J. Acoust. Soc. Am.
Volume 129, Issue 4, pp. 2438-2438, 2011.

\item {} 
SA Eyerly, TD Bahnson, JI Koontz, DP Bradway, DM Dumont, GE Trahey,
PD Wolf. Intracardiac Acoustic Radiation Force Impulse Imaging: A
Novel Imaging Method for Intraprocedural Evaluation of Radiofrequency
Ablation Lesions. Heart rhythm: the official journal of the Heart
Rhythm Society. 1 November 2012, volume 9 issue 11 Pages 1855-1862.

\item {} 
PJ Hollender, DP Bradway, PD Wolf, R Goswami, GE Trahey. Intracardiac
Acoustic Radiation Force Impulse (ARFI) and Shear Wave Imaging in
Pigs with Focal Infarctions. Transactions on Ultrasonics,
Ferroelectrics, and Frequency Control. August, 2013.

\item {} 
V Patel, JJ Dahl, DP Bradway, JR Doherty, SY Lee, SW Smith. Acoustic
Radiation Force Impulse Imaging (ARFI) on an IVUS Circular Array.
Ultrason Imaging. April, 2014 36: 98-111.

\item {} 
SA Eyerly, TD Bahnson, JI Koontz, DP Bradway, DM Dumont, GE Trahey,
PD Wolf. Contrast in Intracardiac Acoustic Radiation Force Impulse
Images of Radiofrequency Ablation Lesions. Ultrason Imaging. April,
2014. 36: 133-148.

\end{itemize}


\subsection{Abstracts and Proceedings}
\label{resume:abstracts-and-proceedings}\begin{itemize}
\item {} 
DP Bradway, SJ Hsu, BJ Fahey, JJ Dahl, TC Nichols, GE Trahey.
Transthoracic Cardiac Acoustic Radiation Force Impulse Imaging: A
Feasibility Study. IEEE Ultrasonics Symposium (IUS), 2007.

\item {} 
BJ Fahey, RC Nelson, SJ Hsu, DP Bradway, DM Dumont, GE Trahey. In
Vivo Acoustic Radiation Force Impulse Imaging of Abdominal Lesions.
IEEE Ultrasonics Symposium (IUS), 2007.

\item {} 
DP Bradway, BJ Fahey, RC Nelson, GE Trahey. ARFI imaging of abdominal
ablation and liver lesion biopsy. International Symposium on
Ultrasonic Imaging and Tissue Characterization, 2009.

\item {} 
DB Husarik, RC Nelson, DP Bradway, BJ Fahey, KR Nightingale, GE
Trahey. First Clinical Experience with Sonographic Elastography of
the Liver Using Acoustic Radiation Force Impulse (ARFI) Imaging. RSNA
2009.

\item {} 
RC Nelson, DP Bradway, BJ Fahey, GE Trahey. Future Application of
Ultrasound: Acoustic Radiation Force Impulse (ARFI) Imaging. AIUM
2009.

\item {} 
DP Bradway, BJ Fahey, RC Nelson, GE Trahey. Recent Clinical Results
of Acoustic Radiation Force Impulse Imaging of Abdominal Ablation.
International Tissue Elasticity Conference 2009.

\item {} 
SJ Hsu, DP Bradway, RR Bouchard, PJ Hollender, PD Wolf, GE Trahey.
Parametric pressure-volume analysis and acoustic radiation force
impulse imaging of left ventricular function. IEEE Ultrasonics
Symposium (IUS), 2010.

\item {} 
DP Bradway, SJ Hsu, PD Wolf, GE Trahey. Acoustic Radiation Force
Impulse Imaging of Acute Myocardial Ischemia and Infarct.
International Symposium on Ultrasonic Imaging and Tissue
Characterization, 2010.

\item {} 
DP Bradway, SJ Hsu, PD Wolf, GE Trahey. Transthoracic Acoustic
Radiation Force Impulse Imaging of Cardiac Function. International
Tissue Elasticity Conference 2010.

\item {} 
PJ Hollender, RR Bouchard, SJ Hsu, DP Bradway, PD Wolf, GE Trahey.
Intracardiac measurements of elasticity using Acoustic Radiation
Force Impulse (ARFI) methods: Temporal and spatial stability of shear
wave velocimetry. IEEE Ultrasonics Symposium (IUS), 2010.

\item {} 
DP Bradway, SJ Rosenzweig, JR Doherty, D Hyun, GE Trahey. Recent
Results and Advances in Transthoracic Cardiac Acoustic Radiation
Force Impulse Imaging. International Symposium on Ultrasonic Imaging
and Tissue Characterization, 2011.

\item {} 
BC Byram, DM Gianantonio, DP Bradway, D Hyun, M Jakovljevic, AL
Crowley, HW Kim, M Parker, R Kim, R Judd, GE Trahey. Direct in vivo
Myocardial Infarct Visualization Using 3D Ultrasound and Passive
Strain Contrast. International Tissue Elasticity Conference 2011.

\item {} 
BC Byram, DP Bradway, M Jakovljevic, D Gianantonio, D Hyun, AL
Crowley, H Kim, L Van Assche, M Parker, R Kim, R Judd, G Trahey.
Direct In Vivo Myocardial Infarct Visualization Using 3D Ultrasound
and Passive Strain Contrast. IEEE Ultrasonics Symp. 2011.

\item {} 
DP Bradway, PJ Hollender, R Goswami, PD Wolf, GE Trahey.
Transthoracic Cardiac Acoustic Radiation Force Impulse Imaging: in
vivo Feasibility, Methods, and Initial Results. International
Symposium on Ultrasonic Imaging and Tissue Characterization, 2012.

\item {} 
PJ Hollender, DP Bradway, R Goswami, PD Wolf, GE Trahey. Acoustic
radiation force techniques for imaging cardiac infarct in vivo:
methods and initial results, International Symposium on Ultrasonic
Imaging and Tissue Characterization, 2012.

\item {} 
DP Bradway, PJ Hollender, R Goswami, PD Wolf, GE Trahey. Feasibility
and Safety of Transthoracic Cardiac Acoustic Radiation Force Impulse
Imaging Methods. 2012 IEEE Ultrasonics Symposium.

\item {} 
SA Eyerly, T Bahnson, J Koontz, DP Bradway, DM Dumont, GE Trahey, PD
Wolf. Confirmation of Cardiac Radiofrequency Ablation Treatment Using
Intra-Procedure Acoustic Radiation Force Impulse Imaging, 2012 IEEE
Ultrasonics Symposium.

\item {} 
PJ Hollender, DP Bradway, PD Wolf, Robi Goswami, Gregg Trahey.
Intracardiac ARF-driven Shear Wave Velocimetry to Estimate Regional
Myocardial Stiffness and Contractility in Pigs with Focal
Infarctions. 2012 IEEE Ultrasonics Symposium.

\item {} 
R Goswami, DP Bradway, J Kisslo, GE Trahey. Novel Application of
Acoustic Radiation Force Impulse Imaging in Transthoracic
Echocardiography. 2013 American College of Cardiology 62nd Annual
Scientific Session.

\item {} 
V Patel, JJ Dahl, DP Bradway, JR Doherty, S Smith. Acoustic Radiation
Force Impulse Imaging (ARFI) on an IVUS Circular Array. 2013 IEEE
UFFC Symposium.

\item {} 
DP Bradway, MJ Pihl, A Krebs, BG Tomov, CS Kjaer, SI Nikolov, JA
Jensen. Real-time GPU implementation of transverse oscillation vector
velocity flow imaging. 2014 SPIE Medical Imaging.

\end{itemize}



\renewcommand{\indexname}{Index}
\printindex
\end{document}
